\usepackage{tabularx}
\usepackage{booktabs}

\usepackage{animate}

\usepackage{verbatim}
\usepackage{comment}
\usepackage{listings}
%%\usepackage{inconsolata}
%
\lstset{
  basicstyle=\ttfamily,
  breaklines=true,
%  breakatwhitespace=true,
%  breakindent=2ex,
% postbreak=\raisebox{0ex}[0ex][0ex]{\ensuremath{\hookrightarrow\space}}
}

\usepackage{amsmath}
\usepackage{amssymb}% for \mathbb
\usepackage{mathtools}
\usepackage{bm}
\usepackage[super]{nth}

\DeclarePairedDelimiter\abs{\lvert}{\rvert}%
\DeclarePairedDelimiter\norm{\lVert}{\rVert}%

% Swap the definition of \abs* and \norm*, so that \abs
% and \norm resizes the size of the brackets, and the 
% starred version does not.
\makeatletter
\let\oldabs\abs
\def\abs{\@ifstar{\oldabs}{\oldabs*}}
%
\let\oldnorm\norm
\def\norm{\@ifstar{\oldnorm}{\oldnorm*}}
\makeatother

\newcommand{\dd}{\mathop{}\!\mathrm{d}}

\newcommand{\vecval}[1]{\bm{#1}}